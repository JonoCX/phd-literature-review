\documentclass{llncs2e/llncs}

%% additional packages here
\usepackage[utf8]{inputenc}
\usepackage{eurosym}

\usepackage{graphicx}
\usepackage{float}

\usepackage[fleqn]{amsmath}
\usepackage{tabularx}
\usepackage{tabulary}
\usepackage{mathpartir}
\usepackage{amssymb}
\usepackage{amsmath}
\usepackage{amsfonts}
\usepackage{subfiles}
\usepackage{soul}

%% additional packages
\usepackage{color}
\usepackage{xcolor}
% shaded frameboxes
\usepackage{framed}
\definecolor{shadecolor}{rgb}{0.9,0.9,0.9}
\definecolor{Orange}{rgb}{1,0.5,0}

\usepackage{mathtools}
\usepackage{cite}
\usepackage{epsfig}
\usepackage{epstopdf}
\usepackage{algorithm}
\usepackage{algorithmic}
\usepackage{caption}
\usepackage[pdftex]{hyperref}
\hypersetup{%
%pdftitle={\myTitle}, %
%pdfauthor={blind submission}, %
%pdfkeywords={\programname},%
%bookmarksnumbered, %
%pdfstartview={c}, %
colorlinks,%
citecolor=black, %
filecolor=black, %
linkcolor=black, %
urlcolor=black}

%\usepackage{blindtext}
%\usepackage{url}
%\usepackage{cite}
%\usepackage{hyperref}
%\usepackage{amsmath}
%\usepackage{amsfonts}
%\usepackage{amssymb}
%\hypersetup{
%colorlinks=true,
%citecolor=blue,
%linkcolor=black,
%urlcolor=black
%}
%\usepackage[numbers]{natbib}


\newcommand{\authornote}[2] {
    \begin{center}
        \framebox{
            \colorbox{yellow!40}{\begin{minipage}[t]{0.9\linewidth}
                \raggedright  \textbf{[#1]}~ \scriptsize #2 \normalsize
            \end{minipage}}
    }
    \end{center}
}

\newcommand{\structurenote}[2] {
    \begin{center}
        \framebox{
            \colorbox{purple!40}{\begin{minipage}[t]{0.9\linewidth}
                \raggedright  \textbf{[#1]}~ \scriptsize #2 \normalsize
            \end{minipage}}
    }
    \end{center}
}

% editing macros (and some cheats)

\newcommand{\lf}{\mbox{~}\\}
\newcommand{\todo}[1]{\textsf{\textbf{\textcolor{Orange}{[[#1]]}}}}
\newcommand{\todoreply}[1]{\textsf{\textbf{\textcolor{purple}{[[#1]]}}}}
\newcommand{\longpage}{\enlargethispage{\baselineskip}}


%% Define program environment for code
\newenvironment{program}
{%
     \framed \scriptsize %
}
{%
     \normalsize \endframed  %
}

\floatname{algorithm}{Procedure}
\renewcommand{\algorithmicrequire}{\textbf{Input:}}
\renewcommand{\algorithmicensure}{\textbf{Output:}}

%% Define subsection depth
\setcounter{secnumdepth}{3}


\title{PhD Literature Review}
\author{Jonathan Carlton}
\institute{School of Computer Science \\ University of Manchester \\
\email{jonathan.carlton@postgrad.manchester.ac.uk}
}

\begin{document}
  \maketitle

  \section{Introduction}

  \section{Interaction}
  In \cite{atterer2006knowing} a monitoring system for web-based interactions is
  defined -- called UsaProxy. By requesting the users of the system to re-route
  all of their connections through a proxy server, HTML pages are modified with
  JavaScript tracking code before they are delivered to the user. The code
  collects data on mouse movements, keyboard input, along with other,
  fine-grained interaction metrics.

  The capture solution presented above, in \cite{atterer2006knowing}, is
  modified in \cite{apaolaza2015longitudinal} to allow deployment by adding
  JavaScript code to the web pages rather than requiring users to set their
  browser to re-route all connections through a proxy server. Data; low-level
  mouse movements, clicks, and keystokes, in this experiement are recorded from
  a high-traffic website continously for two years. They find that users, rather
  than interacting with the website quicker as they become more familiar, have
  increased periods of mouse inactivity. Continually, the users also spend
  more time on the website as they become more familiar. And finally, they
  find that there is no need to collect specific information about users, such
  as any disabilities they may have, as their problems can be indentified through
  emerging behaviours in the experiements \cite{apaolaza2013understanding}.

  Probematic situations encountered by users with visual impairments and the tactics
  they employ to overcome them are explored in \cite{vigo2013evaluating}. Through
  developing several algorithms, and packaging them together into a web-usage
  monitoring tool, the employed tactics are identified and isolated automatically,
  in mouse and keyboard data, and treated as markers to infer the user is having
  an issue. In \cite{vigo2013coping}, more detail is presented about the particular
  type of tactics and an expansion on the analysis process by going deeper into
  the tactics the users employed and how they react to problems (do they give up
  or carry on).

  WevQuery, a scalable system to query user interaction logs collected in \cite{
  apaolaza2015longitudinal, vigo2013evaluating, vigo2013coping}, is presented in
  \cite{apaolaza2017wevquery}. It is a hypothesis testing interface with the aim
  of better understanding user behaviour stored in the data collected (mouse clicks,
  movement, etc.).

  In \cite{gledson2016combining, bull2016combining}, a fully-fledged, desktop
  application with an aim of trying to detect mild cognitive impairment in older
  computer users, through their interactions with the computer, is presented.
  The monitoring system collects data on operating system events, web browsers,
  and applications. Furthermore, mouse movements are collected but the complexity
  is reduced by only recording dragging movements and the time periods between
  clicks. They have early evidence that this is a promising method to detect
  cognitive impairments.

  The authors in \cite{kodagoda2017using} are successfully able to infer the
  reasoning behind analysts' decision making from low-level user interaction logs.
  They detail some information about their data processsing techniques; converting
  features to numeric expressions -- a fairly common practise -- and a set of
  categorical labels to a set of integers which worked for Random Forest and
  Hidden Markov models (scale-invariant models) but not for an SVM, so the integers
  were standardised to have a mean of zero and a variance of one. To test their
  models, they used a control model (no information classifier) which provided
  the baseline for their models to beat.

  In \cite{fu2017your} the aim is to predict user intention from mouse movements,
  clicks, and positioning -- among other features -- with the focus on understanding
  what the user intended when they triggered an interaction and then, from this
  data, attempting to predict the next action. Two classification models are built;
  a probability model and a machine learning model (SVM) model. The probability model
  uses the previous $k$ activities of a user to predict the next activity and a second
  (probability) model considers the time duration of the previous $k$ activities. The
  SVM model is trained on the $k$ most recent user activities, the time duration of those
  activities, and descriptive statisitics about the mouse interations within a set window
  $W$. In their results, they find that both model types have a similar performance in their
  experiments and to achieve the maximum accuracy -- for predicting future interactions
  from historical actions -- the combination of both models, into a multimodal approach,
  achieves the best results (a mean accuracy of 69\%). With regards of variable results,
  they test a range of different values for both $k$ and $W$ and conclude that neither
  can be too large; the best value for $k$ is 3 and when increasing the value of $W$, they
  find that it introduces a large amount of noise.

  A system that can assist a user in an answer discovery task is presented in
  \cite{dabek2017grammar}. Various options are presented to the user in the form
  of a graph, they are then able to traverse the graph to find an answer to their
  question. Throughout the process, an assistive tool provides suggestions on additional
  variables that could help their search (the number of bedrooms when searching for a
  house value, for example) and suggest a new path in their search. While not much can
  be taken from this paper, they make a general point about determining a $k$ value
  in the $k$-means algorithm when there is some uncertainty about the optimal value
  to use; test a range of values between one and the square root of the size of the
  data divided by two. Adding this upper bound prevents over-fitting the value of
  $k$.

  In \cite{gotz2008empirical} the authors attempt to understand a users' analytical
  goal and reasoning through high-level interaction metrics collected from a controlled
  study. They had two main aims in the study; determine common structures of visual
  analysis behaviour and characterise those structures with regards to what they
  represent and how they impact a users' performance. In their results and evaluation,
  they were able to find two behaviour structures of visual analysis activity:
  patterns (short, ad-hoc sequences of visual actions performed iteratively through
  the task) and trails (chains of user visual interaction activity leading to insight).

  A new family of interactions to extend and enrich the input experience for
  users of smartphones are introducted in \cite{zhang2016beyond}. Various additional
  inputs for a smartphone are presented, these include the ability to detect in-pocket
  and on-table touches through the detection of different physical interactions
  with the smartphone. The authors use both machine learning (classification) and
  rule-based (segementation) approaches to analyse characteristics of the interactions
  and determine the action the user is attempting to perform. The rule-based
  approach required no specific training for the users of the smartphone due to looking
  for particular patterns and spikes in the microphone recording, accelerometer and
  gyroscope on the phone, however, the machine learning method (kNN) did. They
  adopted a user-independent model for the implementation of the kNN, which applies
  a pre-collected training data set and does not require any data collection
  from each user. In their evaluation, they found that the machine learning
  approach achieved a range of results between 70\% and 90\% accuracy for the recognition of
  interactions and the rule-based interactions can achieve over 92\% accuracy.

  \section{Engagement}
  A broad review of measuring and defining user engagement in a range of scenarios
  is presented in \cite{brown_glancy}. The focus is on the understanding of initial
  reactions to media-based content and what engagement means in this context. They
  find that if the audience is emotionally invested in the content then their
  level of engagement is subsequently higher.

  In \cite{jennett2008measuring}, the authors perform an investigation to test
  if immersion can be defined quantitatively through experiments. They devise
  three experiments; switching from an immersive to a non-immersive task, changes
  in eye movements during an immersive task, and measuring the effect of an externally
  imposed pace of interaction which alters the flow of the participant. Immersion
  is well defined here and has three features; lack of awareness of time, loss of
  awareness of the real world, and a sense of being in the task environment (a video
  game). They find that immersion can be measured subjectively, through questionnaires
  given to the participants before and after the event, and objectively, through
  task completion and eye movement. The authors apply Spearman's Rank-Order correlation
  on the mouse click data (the mean number of mouse clicks vs the mean number of
  fixations on the non-immersive conidition), relying heavily on this in their
  analysis.

  \section{Sequential Data Mining}
  A review of machine learning approaches, specific to a sequential data analysis
  context, is presented in \cite{dietterich2002machine}. There is a primary focus
  on text- and natural langauge-based solutions and the paper presents algorithms
  such as; sliding window methods, Hidden Markov models, and graph transformer networks.
  The point is made that these approaches are not exclusive to text or natural language
  solutions and have a place in other types of sequential data analysis. An example of these
  approaches being used in other domains is shown in \cite{kodagoda2017using} where
  Hidden Markov models are used in an interaction data context.

  Using annotated sensor data, collected in smart home environments, in
  \cite{carolis2015incremental} the authors aimed to describe how process mining
  can be used to learn users' daily routines. Through utilizing a First-Order learning
  approach (symbolized reasoning in which each sentence, or statement, is broken
  down into a subject and predicate), they propose a solution that can automatically
  learn from daily routines from examples of other people's behaviour. They find,
  through experimentation with toy and publicly available datasets, that their approach
  is effectively able to learn and model the daily routines of people within smart
  home environments. In addition to their work, the authors provide definitions
  for three terms: activity recognition; aims at identifying the occurence of an
  acitivity based on suitable combinations of events, task modelling; aims at
  identifying which combinations of events determine the occurence of an activity,
  and process mining; starts from the activities performed by some agent to carry
  out a given process and identifies the valid patterns of activities that support
  that process. Finally, they also make the point that Hidden Markov model-based
  approaches are more suitable for representing sequential activities from a stream
  of sensor events -- this backs up the point made previously about the various
  applications the models can have.

  \bibliographystyle{apalike}
  \bibliography{references}
\end{document}
