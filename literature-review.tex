\documentclass{llncs2e/llncs}

%% additional packages here
\usepackage[utf8]{inputenc}
\usepackage{eurosym}

\usepackage{graphicx}
\usepackage{float}

\usepackage[fleqn]{amsmath}
\usepackage{tabularx}
\usepackage{tabulary}
\usepackage{mathpartir}
\usepackage{amssymb}
\usepackage{amsmath}
\usepackage{amsfonts}
\usepackage{subfiles}
\usepackage{soul}

%% additional packages
\usepackage{color}
\usepackage{xcolor}
% shaded frameboxes
\usepackage{framed}
\definecolor{shadecolor}{rgb}{0.9,0.9,0.9}
\definecolor{Orange}{rgb}{1,0.5,0}

\usepackage{mathtools}
\usepackage{cite}
\usepackage{epsfig}
\usepackage{epstopdf}
\usepackage{algorithm}
\usepackage{algorithmic}
\usepackage{caption}
\usepackage[pdftex]{hyperref}
\hypersetup{%
%pdftitle={\myTitle}, %
%pdfauthor={blind submission}, %
%pdfkeywords={\programname},%
%bookmarksnumbered, %
%pdfstartview={c}, %
colorlinks,%
citecolor=blue, %
filecolor=black, %
linkcolor=black, %
urlcolor=black}

%\usepackage{blindtext}
%\usepackage{url}
%\usepackage{cite}
%\usepackage{hyperref}
%\usepackage{amsmath}
%\usepackage{amsfonts}
%\usepackage{amssymb}
%\hypersetup{
%colorlinks=true,
%citecolor=blue,
%linkcolor=black,
%urlcolor=black
%}
%\usepackage[numbers]{natbib}


\newcommand{\authornote}[2] {
    \begin{center}
        \framebox{
            \colorbox{yellow!40}{\begin{minipage}[t]{0.9\linewidth}
                \raggedright  \textbf{[#1]}~ \scriptsize #2 \normalsize
            \end{minipage}}
    }
    \end{center}
}

\newcommand{\structurenote}[2] {
    \begin{center}
        \framebox{
            \colorbox{purple!40}{\begin{minipage}[t]{0.9\linewidth}
                \raggedright  \textbf{[#1]}~ \scriptsize #2 \normalsize
            \end{minipage}}
    }
    \end{center}
}

% editing macros (and some cheats)

\newcommand{\lf}{\mbox{~}\\}
\newcommand{\todo}[1]{\textsf{\textbf{\textcolor{Orange}{[[#1]]}}}}
\newcommand{\todoreply}[1]{\textsf{\textbf{\textcolor{purple}{[[#1]]}}}}
\newcommand{\longpage}{\enlargethispage{\baselineskip}}


%% Define program environment for code
\newenvironment{program}
{%
     \framed \scriptsize %
}
{%                
     \normalsize \endframed  %
}

\floatname{algorithm}{Procedure}
\renewcommand{\algorithmicrequire}{\textbf{Input:}}
\renewcommand{\algorithmicensure}{\textbf{Output:}}

%% Define subsection depth
\setcounter{secnumdepth}{3}
 



\title{PhD Literature Review}
\author{Jonathan Carlton}
\institute{School of Computer Science \\ University of Manchester \\
\email{jonathan.carlton@postgrad.manchester.ac.uk}
}

\begin{document}
  \maketitle

  \section{Introduction}

  \section{Interaction}
  In \cite{atterer2006knowing} a monitoring system for web-based interactions is
  defined -- called UsaProxy. By requesting the users of the system to re-route
  all of their connections through a proxy server, HTML pages are modified with
  JavaScript tracking code before they are delivered to the user. The code
  collects data on mouse movements, keyboard input, along with other,
  fine-grained interaction metrics.

  The capture solution presented above, in \cite{atterer2006knowing} is
  modified in \cite{apaolaza2015longitudinal} to allow deployment by adding
  JavaScript code to the web pages rather than requiring users to set their
  browser to re-route all connections through a proxy server. Data; low-level
  mouse movements, clicks, and keystokes, in this experiement are recorded from
  a high-traffic website continously for two years. They find that users, rather
  than interacting with the website quicker as they become more familiar, have
  increased periods of mouse inactivity. Continually, the users also spend
  more time on the website as they become more familiar. And finally, they
  find taht there is no need to collect specific information about users, such
  as any disabilities they may have, as their problems can be indentified through
  emerging behaviours in the experiements \cite{apaolaza2013understanding}.

  Probematic situations encountered by users with visual impairments and the tactics
  they employ to overcome them are explored in \cite{vigo2013evaluating}. Through
  developing several algorithms, and packaging them together into a web-usage
  monitoring tool, the employed tactics are identified and isolated automatically,
  in mouse and keyboard data, and treated as markers to infer the user is having
  an issue. In \cite{vigo2013coping}, more detail is presented about the particular
  type of tactics and an expansion on the analysis process by going deeper into
  the tactics the users employed and how they react to problems (do they give up
  or carry on).

  In \cite{gledson2016combining, bull2016combining}, a fully-fledged, desktop
  application with an aim of trying to detect mild cognitive impairment in older
  computer users, through their interactions with the computer, is presented.
  The monitoring system collects data on operating system events, web browsers,
  and applications. Furthermore, mouse movements are collected but the complexity
  is reduced by only recording dragging movements and the time periods between
  clicks. They have early evidence that this is a promising method to detect
  cognitive impairments.


  \section{Engagement}
  A broad review of measuring and defining user engagement in a range of scenarios
  is presented in \cite{brown_glancy}. The focus is on the understanding of initial
  reactions to media-based content and what engagement means in this context. They
  find that if the audience is emotionally invested in the content then their
  level of engagement is subsequently higher.

  In \cite{jennett2008measuring}, the authors perform an investigation to test
  if immersion can be defined quantitatively through experiments. They devise
  three experiments; switching from an immersive to a non-immersive task, changes
  in eye movements during an immersive task, and measuring the effect of an externally
  imposed pace of interaction which alters the flow of the participant. Immersion
  is well defined here and has three features; lack of awareness of time, loss of
  awareness of the real world, and a sense of being in the task environment (a video
  game). They find that immersion can be measured subjectively, through questionnaires
  given to the participants before and after the event, and objectively, through
  task completion and eye movement. The authors apply Spearman's Rank-Order correlation
  on the mouse click data (the mean number of mouse clicks vs the mean number of
  fixations on the non-immersive conidition), relying heavily on this in their
  analysis.

  \section{Sequential Data Mining (plus others(?))}

  \bibliographystyle{apalike}
  \bibliography{references}
\end{document}
