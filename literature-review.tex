\documentclass{llncs2e/llncs}

%% additional packages here
\usepackage[utf8]{inputenc}
\usepackage{eurosym}

\usepackage{graphicx}
\usepackage{float}

\usepackage[fleqn]{amsmath}
\usepackage{tabularx}
\usepackage{tabulary}
\usepackage{mathpartir}
\usepackage{amssymb}
\usepackage{amsmath}
\usepackage{amsfonts}
\usepackage{subfiles}
\usepackage{soul}

%% additional packages
\usepackage{color}
\usepackage{xcolor}
% shaded frameboxes
\usepackage{framed}
\definecolor{shadecolor}{rgb}{0.9,0.9,0.9}
\definecolor{Orange}{rgb}{1,0.5,0}

\usepackage{mathtools}
\usepackage{cite}
\usepackage{epsfig}
\usepackage{epstopdf}
\usepackage{algorithm}
\usepackage{algorithmic}
\usepackage{caption}
\usepackage[pdftex]{hyperref}
\hypersetup{%
%pdftitle={\myTitle}, %
%pdfauthor={blind submission}, %
%pdfkeywords={\programname},%
%bookmarksnumbered, %
%pdfstartview={c}, %
colorlinks,%
citecolor=black, %
filecolor=black, %
linkcolor=black, %
urlcolor=black}

%\usepackage{blindtext}
%\usepackage{url}
%\usepackage{cite}
%\usepackage{hyperref}
%\usepackage{amsmath}
%\usepackage{amsfonts}
%\usepackage{amssymb}
%\hypersetup{
%colorlinks=true,
%citecolor=blue,
%linkcolor=black,
%urlcolor=black
%}
%\usepackage[numbers]{natbib}


\newcommand{\authornote}[2] {
    \begin{center}
        \framebox{
            \colorbox{yellow!40}{\begin{minipage}[t]{0.9\linewidth}
                \raggedright  \textbf{[#1]}~ \scriptsize #2 \normalsize
            \end{minipage}}
    }
    \end{center}
}

\newcommand{\structurenote}[2] {
    \begin{center}
        \framebox{
            \colorbox{purple!40}{\begin{minipage}[t]{0.9\linewidth}
                \raggedright  \textbf{[#1]}~ \scriptsize #2 \normalsize
            \end{minipage}}
    }
    \end{center}
}

% editing macros (and some cheats)

\newcommand{\lf}{\mbox{~}\\}
\newcommand{\todo}[1]{\textsf{\textbf{\textcolor{Orange}{[[#1]]}}}}
\newcommand{\todoreply}[1]{\textsf{\textbf{\textcolor{purple}{[[#1]]}}}}
\newcommand{\longpage}{\enlargethispage{\baselineskip}}


%% Define program environment for code
\newenvironment{program}
{%
     \framed \scriptsize %
}
{%
     \normalsize \endframed  %
}

\floatname{algorithm}{Procedure}
\renewcommand{\algorithmicrequire}{\textbf{Input:}}
\renewcommand{\algorithmicensure}{\textbf{Output:}}

%% Define subsection depth
\setcounter{secnumdepth}{3}


\title{PhD Literature Review}
\author{Jonathan Carlton}
\institute{School of Computer Science \\ University of Manchester \\
\email{jonathan.carlton@postgrad.manchester.ac.uk}
}

\begin{document}
  \maketitle

  \section{Introduction}

  \section{Interaction}
  In \cite{atterer2006knowing} a monitoring system for web-based interactions is
  defined -- called UsaProxy. By requesting the users of the system to re-route
  all of their connections through a proxy server, HTML pages are modified with
  JavaScript tracking code before they are delivered to the user. The code
  collects data on mouse movements, keyboard input, along with other,
  fine-grained interaction metrics.

  The capture solution presented above, in \cite{atterer2006knowing} is
  modified in \cite{apaolaza2015longitudinal} to allow deployment by adding
  JavaScript code to the web pages rather than requiring users to set their
  browser to re-route all connections through a proxy server. Data; low-level
  mouse movements, clicks, and keystokes, in this experiement are recorded from
  a high-traffic website continously for two years. They find that users, rather
  than interacting with the website quicker as they become more familiar, have
  increased periods of mouse inactivity. Continually, the users also spend
  more time on the website as they become more familiar. And finally, they
  find taht there is no need to collect specific information about users, such
  as any disabilities they may have, as their problems can be indentified through
  emerging behaviours in the experiements \cite{apaolaza2013understanding}.

  \section{Engagement}

  \section{Sequential Data Mining (plus others(?))}

  \bibliographystyle{apalike}
  \bibliography{references}
\end{document}
